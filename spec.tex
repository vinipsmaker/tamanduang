\documentclass[12pt, a4paper]{article}

\usepackage[utf8]{inputenc}
\usepackage[brazilian]{babel}
\usepackage[T1]{fontenc}

\usepackage{hyperref}

\title{Tamanduang}
\author{Vinícius dos Santos Oliveira}
\date{\today}

\begin{document}

\maketitle
\pagebreak
\tableofcontents
\pagebreak

\section{Introdução}

Tamanduang é uma linguagem imperativa, com tipagem estática. Nenhuma função
\texttt{main} é necessária, pois o programa é avaliado e as instruções e
expressões são executadas na ordem em que aparecem no código. A linguagem adota
um modelo de avaliação gulosa.

Em Tamanduang, a declaração de uma variável e sua inicialização devem ser feitas
em etapas diferentes, mas todas as variáveis são implicitamente inicializadas
para algum valor. Esse valor inicial é melhor documentado na referência de cada
tipo que segue no texto, mas o princípio usado para a decisão desse valor
inicial é que a variável deve ser usada causando o mínimo de erros, enquanto
ainda fornece um valor usável.

Funções são variáveis do tipo \texttt{fn}. A linguagem faz uso de escopo léxico
e expressões lambda não capturam variáveis locais. Variáveis globais são
permitidas, então reaproveitamento de código entre diferentes funções é feito
dessa forma, se declarando variáveis no escopo global. Todas atribuições, assim
como a passagem de argumentos para funções, são feitas por cópia. Entretanto,
variáveis globais ainda são acessadas por referência, então ainda é possível ter
um comportamento similar a característica de \emph{forward declaration} da
linguagem C e fazer uso de funções recursivas.

Como funções são variáveis, não há suporte a sobrecarga de funções.

Em Tamanduang, variáveis só são visíveis após terem sido declaradas e somente no
escopo que foram declaradas. E atribuições são instruções, não operadores.

Toda essa preocupação com a abstração de rotinas foi feito pensado em uma maior
unificação e uniformização de como as diferentes partes da linguagem são
utilizadas. Outro exemplo dessa unificação é a sintaxe de acesso a um elemento
de uma lista e as chamadas de funções, que possuem a mesma sintaxe.

Uma parte podre dessa versão inicial da linguagem também está no suporte a
listas, que se baseia em funções-padrões que estão sempre definidas, mas que só
podem ser implementadas no compilador, pois seria necessário que a linguagem
oferece suporte a programação genérica para permitir a implementação de
abstrações com o mesmo poder expressivo fora do compilador.

A linguagem não fornece suporte a bibliotecas, módulos, pacotes, compilação
separada ou qualquer integração maior para dar suporte a reaproveitamento via
código externo (ou compilado separadamente). Entretanto, um pré-processador da
linguagem C, enquanto a linguagem Tamanduang não evolui, pode ser usado.

Outro princípio que guia o desenvolvimento da linguagem é que ela possa ser
usada, tanto como linguagem interpretada, quanto como linguagem compilada.

O nome Tamanduang é uma brincadeira dos nomes Tamanduá e language.

\subsection{Declaração de variáveis}

Variáveis são declaradas através da seguinte sequência de tokens:

\begin{itemize}
\item Palavra-reservada \emph{var}.
\item Identificador de variável.
\item Caractere \emph{:}.
\item Identificador do tipo.
\item Caractere \emph{;}
\end{itemize}

Exemplo de uma declaração, onde declaramos a variável \emph{foobar} do tipo
\emph{int32}:

\begin{verbatim}
var foobar: int32;
\end{verbatim}

\subsection{Tipos primitivos}

Os tipos primitivos são:

\begin{itemize}
\item \texttt{none}.
\item \texttt{bool}.
\item \texttt{int32}.
\item \texttt{double}.
\item \texttt{uchar}.
\end{itemize}

\subsubsection{none}

Nenhum valor pode ser atribuído a uma variável do tipo \texttt{none}. A maior
utilidade desse tipo é declarar o tipo de retorno de funções que não retornam
nada.

Caso a linguagem desse suporte a \emph{pattern matching}, esse tipo poderia ser
mais útil.

\subsubsection{bool}

Representa um valor booleano. Os valores possíveis são \texttt{true} e
\texttt{false}.

O valor inicial de uma variável \texttt{bool} é \texttt{false}.

O operador unário \texttt{!} pode ser usado para inverter um valor
booleano. O operador não possui efeitos colaterais. Exemplo:

\begin{verbatim}
var x: bool;
x = !x;
\end{verbatim}

O operador binário infixo \texttt{and} realiza conjunção (e-lógico), enquanto o
operador lógico \texttt{or} realiza disjunção (ou-lógico). Esses operadores têm
associatividade da esquerda para a direita e possuem precedência menor que o
operador de negação (i.e. \texttt{!}).

\subsubsection{int32}

Um inteiro com sinal que pode representar números do intervalo fechado definido
por $-2147483648$ e $2147483647$.

O valor inicial de uma variável \texttt{int32} é $0$.

\subsubsection{double}

Um número capaz de representar os valores possíveis da especificação de ponto
flutuante de dupla precisão do padrão \emph{IEEE 754}.

O valor inicial de uma variável \texttt{double} é $0.0$.

\subsubsection{int32 e double}

A lista de operações suportadas pelos tipos \texttt{int32} e \texttt{double} é
especificada na tabela abaixo. Todas as operações são livres de efeitos
colaterais.

\begin{tabular}{| p{1.5cm} | p{2cm} | p{1.2cm} | p{1.2cm} | p{2.7cm} | p{5cm} | }
%\begin{tabular}{| l l l l l l | }
\hline
Operador      & Precedência (menor é maior) & Unário/ binário & Pré-fixado/ pós-fixado/ infixo & Associatividade         & Resultado                                                                               \\
\hline
\hline
-             & 1                           & unário         & pré-fixado                   &                         & O inverso negativo                                                                      \\
\hline
*             & 2                           & binário        & infixo                       & esquerda para a direita & Multiplicação                                                                           \\
\hline
+             & 3                           & binário        & infixo                       & esquerda para a direita & Soma                                                                                    \\
\hline
\textless     & 4                           & binário        & infixo                       & esquerda para a direita & booleano representando se operando da esquerda é menor que operando da direita          \\
\hline
\textgreater  & 4                           & binário        & infixo                       & esquerda para a direita & booleano representando se operando da esquerda é maior que operando da direita          \\
\hline
\textless=    & 4                           & binário        & infixo                       & esquerda para a direita & booleano representando se operando da esquerda é menor ou igual que operando da direita \\
\hline
\textgreater= & 4                           & binário        & infixo                       & esquerda para a direita & booleano representando se operando da esquerda é maior ou igual que operando da direita \\
\hline
==            & 5                           & binário        & infixo                       & esquerda para a direita & booleano representando se operandos possuem o mesmo valor                               \\
\hline
!=            & 5                           & binário        & infixo                       & esquerda para a direita & booleano representando se operandos possuem valores diferentes \\
\hline
\end{tabular}

\subsubsection{uchar}

Armazena um \emph{code point} do padrão \emph{Unicode 7.0}.

O valor inicial de uma variável \texttt{uchar} é \emph{U+0000}.

\subsection{Tipos complexos}

Os tipos complexos são:

\begin{itemize}
\item \texttt{list[T]}.
\item \texttt{string}.
\item \texttt{fn[Args..., Ret]}.
\end{itemize}

\subsubsection{list[T]}

Representa uma sequência de valores do mesmo tipo. Os elementos são acessados
por índice e o primeiro elemento recebe índice $0$, o segundo índice $1$ e assim
por diante. Toda lista, imediatamente após sua declaração, não possui nenhum
elemento. O índice sempre é do tipo \texttt{int32}.

O tipo de cada elemento é decidido no momento da declaração, especificado
através do identificador de tipo presente entre os tokens \texttt{[} e
\texttt{]}. Exemplo:

\begin{verbatim}
var list_of_ints: list[int32];
var list_of_functions_to_bool: list[fn[bool]];
\end{verbatim}

O tamanho da lista pode ser consultado através da função \texttt{size}, que
recebe a lista como argumento e retorna seu tamanho (tipo \texttt{int32}).
Exemplo:

\begin{verbatim}
size(list_of_ints);
\end{verbatim}

Um elemento da lista é acessado através do operador pós-fixado \texttt{()}, que
retorna uma referência ao elemento associado aquele índice. O índice é
especificado entre o caractere \texttt{(} e \texttt{)}. Exemplo:

\begin{verbatim}
var x: int32;
x = list_of_ints(1);
list_of_ints(0) = x;
\end{verbatim}

Elementos podem ser adicionados através da função \texttt{push}, que recebe como
argumentos a lista na qual ela está operando, o índice do elemento a ser
adicionado e o elemento a ser adicionado, respectivamente. O índice passado se
tornará o índice do elemento adicionado e todos os elementos que estavam naquela
posição e em posições seguintes serão movidos, tendo seus respectivos índices
incrementados em $1$. O índice não pode ser menor que $0$ ou maior que o tamanho
da lista, ou o elemento não será adicionado. A função retorna um booleano
representando se o elemento foi adicionado. Exemplo:

\begin{verbatim}
push(list_of_ints, 0, 42);
push(list_of_ints, 3, 1337);
\end{verbatim}

Elementos podem ser removidas através da função \texttt{pop}, que recebe como
argumentos a lista na qual ela está operando e o índice do elemento a ser
removido, respectivamente. Se o índice que não está associado a nenhum elemento
for passado como argumento, a função não remove nenhum elemento. A função
retorna um booleano representando se um elemento foi removido. Exemplo:

\begin{verbatim}
pop(list_of_ints, 0);
\end{verbatim}

A função \texttt{concat} recebe como argumentos duas listas do mesmo tipo e
retorna uma lista que é o resultado da concatenação das duas listas. Exemplo:

\begin{verbatim}
var second_list_of_ints: list[int32];
var result: list[int32];
result = concat(list_of_ints, second_list_of_ints);
\end{verbatim}

\subsubsection{string}

Um alias para \texttt{list[uchar]}.

\subsubsection{fn[Args..., Ret]}

Representa uma função. Uma sequência de tipos separados por vírgulas é
especificada entre os colchetes que seguem a palavra reservada \texttt{fn}. O
último desses tipos especifica o retorno da função, enquanto os outros
especificam os argumentos da função. Ao menos um tipo deve ser especificado para
formar um programa válido. Exemplo:

\begin{verbatim}
var next: fn[int, int];
var print: fn[string, none];
\end{verbatim}

O valor inicial de uma variável desse tipo sempre será uma função que não
realiza nenhuma operação.

A uma dessas variáveis, pode ser atribuído uma closure (i.e. o resultado de uma
expressão lambda) que possua a mesma assinatura que a função. A expressão lambda
em Tamanduang é definida pela palavra reservada \texttt{fn}, seguida pelo token
\texttt{(}, seguido de uma lista de argumentos (último sendo o retorno da
função) que são especificados como declarações de variáveis que não possuem o
caractere \texttt{,} e são separados por vírgula, seguido pelo token \texttt{)},
seguido por um bloco de código que possui seu próprio escopo (definido através
dos token \texttt{\{} e \texttt{\}}. O retorno da função é feito através da
instrução \texttt{return}. Exemplo:

\begin{verbatim}
next = fn(var cur: int32, var ret: int32) {
  ret = cur + 1;
  return;
};
\end{verbatim}

O exemplo anterior fez uso da variável de retorno. Essa variável é sempre
inicializada pelas regras de valor inicial da especificação e pode ser
livremente usada dentro do corpo da função.

\subsection{Expressões}

Parentêses podem ser usados para mudar a precedência.

\subsubsection{Estrutura de controle condicional}

É usada a palavra reservada \texttt{if}, seguido de uma expressão que retorne um
booleano e um bloco de código. Exemplo:

\begin{verbatim}
if 2 < 3 {
  var out: string;
  push(out, 0, 'o');
  push(out, 1, 'k');
  print(out);
};
\end{verbatim}

Um código a ser executado caso a condição avalie para falso também pode ser
especificado, como mostra no exemplo a seguir:

\begin{verbatim}
if 2 < 3 {
  var out: string;
  push(out, 0, 'o');
  push(out, 1, 'k');
  print(out);
} else {
  var out: string;
  push(out, 0, 'n');
  push(out, 1, 'o');
  print(out);
};
\end{verbatim}

\subsubsection{Estrutura de repetição: while}

Similar ao \texttt{if}. Repete o código enquanto a condição for
verdadeira. Exemplo:

\begin{verbatim}
var x: int32;
while x < 2 {
  x = x + 1;
  var out: string;
  push(out, 0, '\n');
  print(out);
};
\end{verbatim}

\subsubsection{Estrutura de repetição: for ... until ...}

Similar ao \texttt{while}, porém um identificador de variável é colocado após
\texttt{for} e um literal do tipo \texttt{int32} é colocado após
\texttt{until}. O código será repetido com a variável assumindo os valores entre
$0$ até o valor especificado (inclusivo). Exemplo:

\begin{verbatim}
for x until 1 {
  var out: string;
  push(out, 0, '\n');
  print(out);
};
\end{verbatim}

\subsection{Entrada e saída}

Funções para entrada:

\begin{itemize}
\item \texttt{read\_int32}: Lê um valor do tipo \texttt{int32}.
\item \texttt{read\_string}: Lê uma palavra.
\end{itemize}

Funções para saída:

\begin{itemize}
\item \texttt{print}: Imprimi os caracteres passados pela sequência.
\item \texttt{println}: Imprimi os caracteres passados pela sequência e imprimi
  um final de linha.
\end{itemize}

\section{Exemplos}

\subsection{Hello World}

\begin{verbatim}
var out: string;
push(out, 0, 'H');
push(out, 1, 'e');
push(out, 2, 'l');
push(out, 3, 'l');
push(out, 4, 'o');
push(out, 5, ' ');
push(out, 6, 'W');
push(out, 7, 'o');
push(out, 8, 'r');
push(out, 9, 'l');
push(out, 10, 'd');
println(out);
\end{verbatim}

\subsection{Fibonacci}

\begin{verbatim}
var fib: fn[int32, int32];
fib = fn(var n: int32, var ret: int32) {
  if n <= 0 {
    return;
  }

  var i: int32;
  var t: int32;

  i = 1;
  n = n - 1;

  for k until n {
    t = i + ret;
    i = ret;
    ret = t;
  };
  return;
};

var n: int32;
n = read_int32();
fib(n);
\end{verbatim}

\end{document}
